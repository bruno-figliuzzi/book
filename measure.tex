\chapter{Basic facts on probability and measure theory}

In this appendix, we recall the main results of measure theory and probability theory. A measure on a set is a systematic way to assign a number to each suitable subset of that set. In this sense, a measure is a generalization of the concepts of length, area, and volume. Technically, a measure is a function that assigns a non-negative real number or +$\infty $ to (certain) subsets of a set $X$. It must assign $0$ to the empty set and be countably additive: if we consider a large subset $Y$ of $X$ that we decompose in smaller disjoint subsets, the measure $Y$ will necessarilly be the sum of the measures of the smaller subsets. Probability theory strongly relies on the notion of measure. Probability theory considers measures that assign to the whole set the size $1$, and considers measurable subsets to be events whose probability is given by the measure. In this appendix, our aim is to recall some basic facts related to measuretheory and probability.


\section{$\sigma $-algebra}

Let $\Omega $ be the \textit{fundamental} set of all possible outcomes of a random experiment. The aim of probability theory is to quantify the occurence of some subsets of $\Omega $, called events. We consider for instance all possible outcome of a dice throw. In this case, the fundamental set $\Omega $ will be constituted by the outcomes
$$
\Omega = \{1, 2, 3, 4, 5, 6 \}.
$$
Events can be defined as subsets of $\Omega $. For instance, the event "the outcome of the dice throw is $5$" simply corresponds to the subset $\{5 \}$. This approach allows us to consider more complicated events. For instance, the event "the outcome of the dice throw is NOT $5$" corresponds to the subset $\{1, 2, 3, 4, 6 \} = \{5\}^c$. Similarly, the event "the outcome of the dice throw is strictly less than $5$" corresponds to the subset $\{1, 2, 3, 4 \}$. In general, we can note than the conjonction or on the contrary the disjonction of events, as well as the negation of events, are events too. In mathematical terms, the set of all events thus verifies the algebraic properties of the $\sigma $- algebra.

\begin{definition}
A $\sigma $-algebra $\mathcal{A}$ on $\Omega $ is a class of subsets of $\Omega $ such that\\
- $\emptyset \in \mathcal{A}$, \\
- If $A \in \mathcal{A}$, then $A^c \in \mathcal{A}$, \\
- For all countable family $\mathcal{I}$, if for all $i \in \mathcal{I}$, $A_i \in \mathcal{A}$, $\cup_{i \in \mathcal{I}} A \in \mathcal{A}$.\\
The set $\Omega $ along with its $\sigma $-algebra $\mathcal{A}$ is called the measurable space $(\Omega, \mathcal{A})$. 
\end{definition} 

\begin{problem}
Show that a $\sigma $-algebra is stable by intersection.
\end{problem}

\begin{problem}
Let $\Omega $ be some fundamental set. Check that $(\Omega, \mathcal{P}(\Omega ))$ is a measurable space, where $\mathcal{P}(\Omega )$ denotes the set of all subsets of $\Omega $.
\end{problem}

For non countable fundamental sets, the $\sigma $-algebra $(\Omega, \mathcal{P}(\Omega ))$ can remain highly complicated. Therefore, one often considers simpler $\sigma $-algebra generated by some class of subsets of $\Omega $. 

\begin{definition}
The $\sigma $-algebra generated by a class $\mathcal{C}$ of subsets of $\Omega $ is the smallest $\sigma $-algebra containing $\mathcal{C}$. In particular, when $\Omega = \mathbb{R}^d$, the $\sigma $-algebra $\mathcal{B}(\mathbb{R}^d)$ generated by the open sets of $\mathbb{R}^d$ is called the Borelien $\sigma $-algebra of $\mathbb{R}^d$.
\end{definition}

\section{Measures and probability}

A measure is simply a functional that associates to each element of a $\sigma $-algebra a positive real number. The area in $\mathbb{R}^2$ is a simple example of a measure defined on the measurable space $(\mathbb{R}^2, \mathcal{B}(\mathbb{R}^2))$.

\begin{definition}
Let $(\Omega, \mathcal{A})$ be a measurable space. A measure on $(\Omega, \mathcal{A})$ is a function $m:\mathcal{A} \rightarrow \mathbb{R}_+ \cup \infty $ such that\\
- $m(\emptyset ) = 0$.\\
- $m$ is $\sigma $-additive, meaning that
$$
m \bigg ( \cup_{i = 1}^{+ \infty } A_i \bigg ) = \sum_{i = 1}^{+\infty }m(A_i)
$$
with $A_i \cap A_j = \emptyset $ if $i \neq j$.
\end{definition}

A measure is said to be finite if the measure of the whole space is finite: $m(\Omega ) < \infty$. In particular, a probability measure is a measure such that the measure of the fundamental space is $m(\Omega ) = 1$. If we go back to our first example of a dice throw, the probability that the results belongs to the set $\Omega = \{1, 2, 3, 4, 5, 6 \}$ is indeed $1$ and the probability measure of each subset is simply interpreted as the probability of the corresponding event.

\begin{problem}
Let $P$ be a probability measure on some measurable space $(\Omega , \mathcal{A})$ and $\{A_i \}$ be some family of events. Show that:\\
- if $A_i \subset A_j$, then $p(A_i) \leq p(A_j)$\\
- $p(\cup_i A_i ) \leq \sum_{i} p(A_i)$.
\end{problem}

An example of measure is provided by the Dirac measure $\delta _x$ associated to $x$:
$$
\delta_x(y) = 1 \text{ if } y = x\text{, } 0 \text{ otherwise.}
$$
Another fundamental example is the indicative function of the subset $A$ of $\Omega $.
$$
\mathrm{1}_A(x) = 1 \text{ if } x \in A \text{, } 0 \text{ otherwise.}
$$ 

\section{Lebesgue measure}

Let $(\mathbb{R}^d, \mathcal{B}(\mathbb{R}^d))$ be the euclidean measurable space of dimension $d$ with its Borel $\sigma$-algebra. A Radon measure on $\mathcal{B}(\mathbb{R}^d)$ is a measure $m$ such that for all bounded subset $B$ of $\mathcal{B}(\mathbb{R}^d)$, $m(B) < \infty$.\\

Among all Radon measures, Lebesgue measures play a particular role. Lebesgue measures are first defined on hypercubes of $\mathbb{R}^d$ to be 
$$
\mu (Q) = (x^0_1 - x^0_0)..(x^d_1 - x^d_0),
$$
where $Q$ is the hypercube $[x^0_0, x^0_1] \times ... \times [x^d_0, x^d_1]$, and can next be generalized to any subset in $\mathcal{B}(\mathbb{R}^d)$. In $\mathbb{R}^3$ (resp. $\mathbb{R}^2$), the Lebesgue measure of a domain is simply its volume (resp. area).\\

One can easily check that the Lebesgue measure has the property to be isometry-invariant. For instance, in the plane $\mathbb{R}^2$, if we translate and/or rotate some domain, its area remains unchanged. In addition, we have the fundamental result:

\begin{theorem}
Let $\nu $ be some Radon measure on $(\mathbb{R}^d, \mathcal{B}(\mathbb{R}^d))$. If $\nu $ is isometry-invariant, then there exist a real number $\lambda > 0$ such that $\nu = \lambda \mu_d $, where $\mu_d$ is the Lebesgues measure on $(\mathbb{R}^d, \mathcal{B}(\mathbb{R}^d))$. 
\end{theorem}

\section{Measurable functions and random variables}

\begin{definition}
Let $f:X \rightarrow Y$ be some function between two measurable spaces $(X, \mathcal{X})$ and $(Y, \mathcal{Y})$. $f$ is said to be measurable if for all element $B$ of the $\sigma$-algebra $\mathcal{Y}$, $f^{-1}(B)$ is an element of the $\sigma$-algebra $\mathcal{X}$.
\end{definition}
In practice, most usual functions are measurable. In particular, all continuous functions from $\mathbb{R}^d$ to $\mathbb{R}^{d'}$ are measurable for the Borel $\sigma$-algebra.\\

In probability theory, a random variable is a measurable function from the fundamental set $\Omega $ of all possible outcomes of some random experiment. As an example, we consider the events constituted by two dice rollings. The function
$$
f: (n_1, n_2) \subset \Omega \times \Omega \rightarrow n_1 + n_1 \in \mathbb{N}
$$
which associates their sum to the results of two dice rollings is a random variable on $\Omega \times \Omega $. 

\section{Probability density}

A direct application of measure theory is the construction of integrals with respect to some measures. A complete exposition of this construction is out of the scope of this appendix, and we refer the readers interested by these topics to the vast literature on the subject. In this paragraph, we briefly consider the case of probabilities defined by a density functional.\\

We consider the measurable space $(\mathbb{R}^d, \mathcal{B}(\mathbb{R}))$. Let $p:\mathbb{R}^d \rightarrow \mathbb{R}$ a non-negative function such that 
$$
\int_{\mathbb{R}^d} p(r) dr = 1,
$$
$dr$ being the Lebesgue measure on $\mathbb{R}^d$. The measure of each Borel set $A$ of $\mathbb{R}^d$ is defined to be
$$
P(A) = \int_{A} p(r) dr.
$$
It is clear that the functional $P:\mathcal{B}(\mathbb{R}^d) \rightarrow \mathbb{R}_+$ is a probability measure. The function $p$ is the density associated to the probability $P$ and can be interpreted as follows:
$$
p(x)dx = P([x, x + dx]).
$$
Let $f$ be a random variable on the mesured space $(\mathbb{R}^d, \mathcal{B}(\mathbb{R}), P)$. The expectation of $f$ is given by
$$
\mathbb{E}[f] = \int_{\mathbb{R}} f(r) p(r) dr.
$$
Similarly, its variance is 
$$
\text{var}[f] = \int_{\mathbb{R}} (f(r) -  \mathbb{E}[f])^2 p(r) dr.
$$

An example of probability density is provided by the uniform law on some interval $[a, b]$ of $\mathbb{R}$. For the uniform law, the probability density is
$$
p(x) = \dfrac{1}{b - a}.
$$
Hence, we find, for all $c$ such that $a \leq x \leq b$, 
$$
P\{X > x \} = \int_{x}^{b} \dfrac{dx}{b - a} = \dfrac{b - x}{b - a}.
$$
Another fundamental example of probability density is provided by the gamma distribution. The gamma distribution is characterized by two parameters, namely the shape parameter $k$ and the scale parameter $\lambda $. Its density is given by 
$$
p(x) = \dfrac{x^{k - 1} \exp(-\frac{x }{\lambda  }}{\Gamma (k) \lambda ^k }.
$$

\begin{problem}
Calculate the expectation and the variance of the gamma law.
\end{problem}


\section{Notes}

The lecture notes of Le Gall~\cite{legall} provide a very good introduction to measure, integration and probability theory.


